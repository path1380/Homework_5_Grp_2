\documentclass{article}
\usepackage[utf8]{inputenc}
\usepackage{amssymb}
\usepackage{amsmath}
\usepackage{float}
\usepackage{epstopdf}
\usepackage{moreverb}
\usepackage{multicol}
\usepackage{listings}
\usepackage{mathrsfs}
\usepackage{graphicx}
\usepackage{cite}
\usepackage{tabularx}
\usepackage{listings}
\usepackage{subcaption}
\newcommand{\R}{\mathbb{R}}
\newcommand{\overbar}[1]{\mkern 1.5mu\overline{\mkern-1.5mu#1\mkern-1.5mu}\mkern 1.5mu}
\newcommand{\pdiff}[2]{\frac{\partial {#1}}{\partial {#2}}}


\bibliographystyle{plain}
\title{APPM 5720 Homework 4}
\author{Wil Boshell, Fortino Garcia, Alex Rybchuk, Parth Thakkar}
\date{February 22, 2018}


\begin{document}

\maketitle

\maketitle

\noindent The main task of this homework is to approximate the transport equation
	\begin{align*}
		u_t + \left( a u\right)_x = 0, \quad x_L \leq x \leq x_R, \quad t > 0,
	\end{align*}
with initial conditions $u(x,0) = f(x)$.
\subsection{Gordon-Hall Mapping}
\noindent In order to use the mapping to compute integrals (and derivatives) we will need to compute the metric, i.e. $r_x, \, r_y, \, s_x, \, s_y$. Recall that the chain rule for a function of two variables requires that:
  \begin{align*}
    \pdiff{u(x(r,s),y(r,s))}{x} & = \pdiff{r}{x}\pdiff{u}{r} + \pdiff{s}{x} \pdiff{u}{s}, \\
    \pdiff{u(x(r,s),y(r,s))}{y} & = \pdiff{r}{y}\pdiff{u}{r} + \pdiff{s}{y} \pdiff{u}{s}.
  \end{align*}
By substituting $u = x$ and $u = y$ into the above equations, we arrive at:
  \begin{align*}
    1 & = \pdiff{x}{x} = \pdiff{r}{x}\pdiff{x}{r} + \pdiff{s}{x} \pdiff{x}{s} = r_x x_r + s_x x_s, \\
    1 & = \pdiff{y}{y}  = \pdiff{r}{y}\pdiff{y}{r} + \pdiff{s}{y} \pdiff{y}{s} = r_y y_r + s_y y_s.  
  \end{align*}
Rewriting this as a matrix system, we arrive at 

  \begin{equation}
    \begin{bmatrix}
      1 & 0 \\
      0 & 1
    \end{bmatrix}
    = 
    \begin{bmatrix}
      r_x & s_x \\
      r_y & s_y
    \end{bmatrix}
    \begin{bmatrix}
      x_r & y_r \\
      x_s & y_s
    \end{bmatrix}.
  \end{equation}
Using the explicit formula for the inverse of a matrix, we immediately solve for the metric via
  \begin{equation}
    \begin{bmatrix}
      r_x & s_x \\
      r_y & s_y
    \end{bmatrix}
    =
    \begin{bmatrix}
      x_r & y_r \\
      x_s & y_s
    \end{bmatrix}^{-1}
    = \frac{1}{x_r y_s - x_s y_r}
    \begin{bmatrix}
      y_s & -y_r \\
      -x_s & x_r
    \end{bmatrix}
    = \frac{1}{J}
    \begin{bmatrix}
      y_s & -y_r \\
      -x_s & x_r
    \end{bmatrix}.
  \end{equation}
where $J = x_r y_s - x_s y_r$. Note that this gives

\subsection{Line Integrals}
We now consider finding the area of an element via line integrals (instead of the previously direct volume integral). To that end, consider the vector field $F(x,y) = (x/2, y/2)$. Then $\nabla \cdot F(x,y) = 1$ and by the divergence theorem:

  \begin{align*}
    V =  \int_\Omega 1 \, dV = \int_\Omega \nabla \cdot F(x,y) \, dV = \int_{\partial \Omega} \nabla F(x,y) \cdot \hat{\textbf{n}} \, dl = \int_{\partial \Omega} \left( \frac{x}{2}n_1 + \frac{y}{2}n_2 \right)\, dl,
  \end{align*}
where $\hat{\textbf{n}} = (n_1,n_2)$. If we consider the reference square, side 1 corresponds to $s = 1$ and $r \in [-1,1]$. The tangent to the curve along this face is clearly $(x_r,y_r)$ so that the normal is
  \begin{align*}
    (-y_r, \, x_r) = J (s_x, \, s_y) \implies \hat{\boldsymbol{n_1}} = \frac{(s_x, \, s_y)}{\sqrt{s_x^2 + s_y^2}}.
  \end{align*}
A similar calculation for the other 3 faces will yield the following set of normals (where the subscript indicates which face it belongs to):

  \begin{align*}
    \hat{\boldsymbol{n_1}} = \frac{(s_x, \, s_y)}{\sqrt{s_x^2 + s_y^2}}, & \quad
    \hat{\boldsymbol{n_2}} = -\frac{(r_x, \, r_y)}{\sqrt{r_x^2 + r_y^2}}, \\
    \hat{\boldsymbol{n_3}} = -\frac{(s_x, \, s_y)}{\sqrt{s_x^2 + s_y^2}}, & \quad
    \hat{\boldsymbol{n_4}} = \frac{(r_x, \, r_y)}{\sqrt{s_x^2 + s_y^2}}. \\
  \end{align*}

% \begin{figure}[H]
%   \centering
%   \includegraphics[scale=0.7]{plots/eig_data.png}
%   \caption{Plot of $\lambda _{max}$ with Increasing $q$}
%   \label{fig:eig_data}
% \end{figure}

% \begin{figure}[H]
%   \centering
%   \begin{minipage}{.4\textwidth}
%     \centering
%     \includegraphics[width=\linewidth]{plots/l2_time_1.png}
%     \label{fig:smallstep}
%   \end{minipage}%
%   \begin{minipage}{.4\textwidth}
%     \centering
%     \includegraphics[width=\linewidth]{plots/l2_time_2.png}
%     \label{fig:medstep}
%   \end{minipage}%
%   \begin{minipage}{.4\textwidth}
%     \centering
%     \includegraphics[width=\linewidth]{plots/l2_time_3.png}
%     \label{fig:bigstep}
%   \end{minipage}%
% \end{figure}
\end{document}


